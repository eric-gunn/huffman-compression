\documentclass{article}
\usepackage{mathtools}
\usepackage{forest}
\usetikzlibrary{decorations.pathreplacing}
\usepackage{microtype}
\title{Huffman Compression}
\author{Eric Gunn}
\date{}%remove date
\setlength\parindent{0pt}%set indent to 0 pt
\begin{document}
\maketitle %render title

Huffman compression is an interesting way to use space efficiently by reducing the number of bits needed to represent a string of text. For example, take the string \textbf{GOOD BOOKKEEPER} and convert it to its ASCII binary representation:
	%pre-compression table
	\begin{center}
		\begin{tabular}{| c | c |}
			\hline
			Character & Binary ASCII Representation\\
			\hline
			G & 01000111\\
			O & 01001111\\
			O & 01001111\\
			D & 01000100\\
			  & 00100000\\
			B & 01000010\\
			O & 01001111\\
			O & 01001111\\
			K & 01001011\\
			K & 01001011\\
			E & 01000101\\
			E & 01000101\\
			P & 01010000\\
			E & 01000101\\
			R & 01010010\\
			\hline
		\end{tabular}
	\end{center}
Each binary ASCII value is one byte (8 bits). Since there are 15 total characters in the string (including the space), we know that the string is 120 bits long. Huffman compression works by using fewer bits to represent the most frequent characters in a string. Here is a table of the characters in the string sorted by frequency in descending order:
	%frequency table
	\begin{center}
		\begin{tabular}{| c | c |}
			\hline
			Character & Frequency\\
			\hline
			O & 4\\
			E & 3\\
			K & 2\\
			B & 1\\
			D & 1\\
			G & 1\\
			P & 1\\
			R & 1\\
			  & 1\\
			\hline
		\end{tabular}
	\end{center}
If the frequency table is taken and put it into a tree, with each node being a sum of its children, the following is the result (here, \underline{\hspace{10pt}} is used to represent a space. The character appears on the left with the frequency or sum of frequencies on the right, separated with a colon.): 
	\begin{center}
		%start tree
		\begin{forest}
		for tree={l=2cm}
			%node oekbdgpr_
			[OEKBDGPR\underline{\hspace{10pt}}:15,for tree=draw,name=oekbd
				%node oek
				[OEK:9,edge label={node[midway,left]{0}},name=oek
					%node o
					[$\underbrace{\text{O:4}}_{00}$,l*=3,edge label={node[midway,left]{0}},name=o]
					%node ek
					[EK:5,l*=2,edge label={node[midway,right]{1}},name=ek
						%node e
						[$\underbrace{\text{E:3}}_{010}$,edge label={node[midway,left]{0}},name=e]
						%node k
						[$\underbrace{\text{K:2}}_{011}$,edge label={node[midway,right]{1}},name=k]
					]
				]
				%node bdgpr_
				[BDGPR\underline{\hspace{10pt}}:6,edge label={node[midway,right]{1}},name=bdgpr
					%node bd
					[BD:2,l*=2,edge label={node[midway,left]{0}},name=bd
						%node b
						[$\underbrace{\text{B:1}}_{100}$,edge label={node[midway,left]{0}},name=b]
						%node d
						[$\underbrace{\text{D:1}}_{101}$,edge label={node[midway,right]{1}},name=d]
					]
					%node gpr
					[GPR\underline{\hspace{10pt}}:4,edge label={node[midway,right]{1}},name=gpr
						%node gp
						[GP:2,edge label={node[midway,left]{0}},name=gp
							%node g
							[$\underbrace{\text{G:1}}_{1100}$,edge label={node[midway,left]{0}},name=g]
							%node p
							[$\underbrace{\text{P:1}}_{1101}$,edge label={node[midway,right]{1}},name=p]
						]
						%node r_
						[R\underline{\hspace{10pt}}:2,edge label={node[midway,right]{1}},name=rspace
							%node r
							[$\underbrace{\text{R:1}}_{1110}$,edge label={node[midway,left]{0}},name=r]
							%node _
							[$\underbrace{\text{\underline{\hspace{10pt}}}:1}_{1111}$,edge label={node[midway,right]{1}},name=space]
						]
					]
				]
			]
		%arrow styling in scope
		\begin{scope}[>=latex]
			%right arrows
			\foreach \start/\dest in {oekbd/bdgpr,bdgpr/gpr,gpr/rspace,rspace/space,oek/ek,ek/k,bd/d,gp/p}{
				\draw [->,dashed] (\start) to [out=south east,in=north] (\dest);
			}
			%left arrows
			\foreach \start/\dest in {oekbd/oek,oek/o,bdgpr/bd,bd/b,gpr/gp,gp/g,rspace/r}{
				\draw [->,dashed] (\start) to [out=south west,in=north] (\dest);
			}
		\end{scope}
		\end{forest}
	\end{center}
Each node in the tree is assigned either a 0 if on the left, or a 1 if on the right. By traversing the tree to its leaves (endpoints) it is found that there is a unique path to each leaf, so each character can now assume a new (and smaller) binary representation.
	%post-compression table
	\begin{center}
		\begin{tabular}{| c | c | c |}
			\hline
			Character & Frequency & Huffman Compression Representation\\
			\hline
			O & 4 & 00\\
			E & 3 & 010\\
			K & 2 & 011\\
			B & 1 & 100\\
			D & 1 & 101\\
			G & 1 & 1100\\
			P & 1 & 1101\\
			R & 1 & 1110\\
			  & 1 & 1111\\
			\hline
		\end{tabular}
	\end{center}
After the compression process, the string is now 45 bits instead of 120 bits, yielding a 62.5\% reduction in space.
\end{document}




























